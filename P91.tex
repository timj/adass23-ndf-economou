% This is the aspauthor.tex LaTeX file
% Copyright 2010, Astronomical Society of the Pacific Conference Series

\documentclass[11pt,twoside]{article}
\usepackage{asp2010}

\resetcounters

\bibliographystyle{asp2010}

\markboth{Economou et al.}{Lessons learned from NDF}

\newcommand{\aspconf}{ASP Conf.\ Ser.}

\begin{document}

\title{Advantages of extensible self-described data formats: Lessons learned from NDF}
\author{Frossie~Economou$^1$, Tim~Jenness$^4$, Malcolm~J.~Currie$^3$,
and David~S.~Berry$^3$
\affil{$^1$National Optical Astronomy Observatory, 950 N.\ Cherry Ave,
Tucson, AZ 85719, USA}
\affil{$^2$Department of Astronomy, Cornell University, Ithaca, NY
  14853, USA}
\affil{$^3$Joint Astronomy Centre, 660 N.\ A`oh\=ok\=u Place, Hilo, HI
96720, USA}
}

\begin{abstract}
  In the context of current discussions about the future of data
  formats, we present an overview of key features of the NDF
  format. In our experience, these features offer advantages for
  archive-side reprocessing and publication of data products, and
  provide valuable lessons learned for future data format designers.
\end{abstract}

\section{Introduction}

The extensible N-Dimensional Data Format
\citep[NDF;][]{1988STARB...2...11C,1993ASPC...52..229W,SGP38} was a
data model layered on top of the Starlink Hierarchical Data System
\citep[HDS;][]{1982QJRAS..23..485D}. During the 1980s it became clear
that a general hierarchical file format without regard for naming
conventions was a recipe for incompatibility and confusion. NDF was
developed in response to this issue and was released in 1987.

NDF contains standardised naming conventions for representing data,
variance and quality masks, handling history and FITS headers, along
with a data access library that can support data sectioning, including
WCS slices. Over the years the format has been extended to support
world-coordinate objects \citep{2001ASPC..238..129B}, data compression
\citep{2008ASPC..394..650C} and provenance tracking
\citep{2009ASPC..411..418J}.

\section{NDF components}

In this section we provide a general overview of the data model
implemented by NDF.

\subsection{Data, Variance, and Quality}

Data arrays, variance components and quality masks are all grouped
into \texttt{ARRAY} structures which also support the concept of
``pixel'' coordinates to allow the NDF to be shifted in space relative
to other NDFs. Along with these N-dimensional array components, title,
data label and data units are explicitly supported metadata.

A mask array is available to allow each pixel in the main data array
to be associated with a particular quality. NDF supports up to 8 bits
of quality, and the bits can be associated with easy-to-remember
string labels, and the individual bits can be activated based on a bit
mask applicable to the whole array.

\subsection{World Coordinates}

World Coordinates are handled by storing a serialized WCS object
generated by the AST library \citep{1998ASPC..145...41W}. AST can
support arbitrary stacked mappings between arbitrary frames and
supports the full range of FITS WCS standards \citep[see
e.g.][]{2012ASPC..461..825B}.

\subsection{History and Provenance}

Rather than being unstructured comment fields, NDF stores history
metadata in explicit data structures with a supporting API. The
history tracks the date, the application name, the user running the
application and the command parameters.

Provenance (determining which files contributed to each step in the
history) can be tracked automatically, by determining which files were
opened by the application, or by explicitly noting which files are
relevant to the output provenance. An integer-based provenance
structure is used for performance reasons (which can be an issue when
there are thousands of entries in the provenance). The full history is
also recorded for each step in the provenance. Provenance tracking is
critical for archive processing where a scientist downloading a
product should be able to determine which observations went into the
product and enables them to regenerate it if so desired. The Canadian
Astronomy Data Centre includes provenance in their archive data model
\citep[CAOM;][]{P25_adassxxii} and the data pipeline for the JCMT
Science Archive \citep{2008ASPC..394..565J} was able to include
provenance information in its products without requiring extensive
instrument-specific modifications.


\subsection{Extensions}

Arbitrary structures can be stored in the \texttt{MORE}
extension. Extensions can be arbitrary HDS structures but it's
recommended that where possible NDFs are used to represent array
data. For example, the SMURF application includes an NDF structure
representing an exposure time image in its extension. NDF applications
can address structures at any position in the file structure. A SMURF
exposure time image can be displayed by referring to
\texttt{file.MORE.SMURF.EXP\_TIME}. Software packages are free to
define their own structures on the understanding that packages that do
not understand will still propagate the structures during processing.

General keyword/value headers are stored in the extension as an array of
80-character FITS headers. This was a pragmatic decision and ensures
easy compatibility with FITS header parsers and conversion of NDF to
and from FITS.

\section{Lessons Learned}

The developers of NDF were aiming for a general purpose astronomical
data format suitable for use across multiple wavelength ranges. They
therefore had to provide features that were genuinely useful without
bloating the specification with features demanded by a niche
community.

Over 25 years of use, the NDF approach to a structured data model has
succeeded in its aim of supporting a rich suite of data reduction and
analysis tools in a consistent manner. It has also proven to be
possible to round-trip to FITS files without losing the complex
organizational structure. It is hoped that these lessons can feed into
discussions on the future direction of FITS as espoused in
\citet{P90_adassxxiii}.

A few lessons have been learned from experience with using NDF. It
became clear that restricting quality to only 8 bits was not a
sustainable solution and a flexible bit count is required. Software
such as the SMURF package \citep{2013MNRAS.430.2545C} now make use of
16-bit quality internally and have to combine bits when exporting to
NDF. Additionally, there was a tendency early on to create overly
complex application-specific extensions rather than trying to re-use
more widely understood structures. Finally, the lack of native table
support in HDS (something added to FITS \citep{1988A&AS...73..365H}
whilst NDF was being developed) has restricted the applicability of
NDF for some applications.

\bibliography{P91}

\end{document}
